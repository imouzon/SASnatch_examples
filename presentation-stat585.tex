
\documentclass[xcolor=dvipsnames,gray,mathserif]{beamer}
\usepackage[]{graphicx}\usepackage[]{color}
%% maxwidth is the original width if it is less than linewidth
%% otherwise use linewidth (to make sure the graphics do not exceed the margin)
\makeatletter
\def\maxwidth{ %
  \ifdim\Gin@nat@width>\linewidth
    \linewidth
  \else
    \Gin@nat@width
  \fi
}
\makeatother

\definecolor{fgcolor}{rgb}{0.345, 0.345, 0.345}
\newcommand{\hlnum}[1]{\textcolor[rgb]{0.686,0.059,0.569}{#1}}%
\newcommand{\hlstr}[1]{\textcolor[rgb]{0.192,0.494,0.8}{#1}}%
\newcommand{\hlcom}[1]{\textcolor[rgb]{0.678,0.584,0.686}{\textit{#1}}}%
\newcommand{\hlopt}[1]{\textcolor[rgb]{0,0,0}{#1}}%
\newcommand{\hlstd}[1]{\textcolor[rgb]{0.345,0.345,0.345}{#1}}%
\newcommand{\hlkwa}[1]{\textcolor[rgb]{0.161,0.373,0.58}{\textbf{#1}}}%
\newcommand{\hlkwb}[1]{\textcolor[rgb]{0.69,0.353,0.396}{#1}}%
\newcommand{\hlkwc}[1]{\textcolor[rgb]{0.333,0.667,0.333}{#1}}%
\newcommand{\hlkwd}[1]{\textcolor[rgb]{0.737,0.353,0.396}{\textbf{#1}}}%

\usepackage{framed}
\makeatletter
\newenvironment{kframe}{%
 \def\at@end@of@kframe{}%
 \ifinner\ifhmode%
  \def\at@end@of@kframe{\end{minipage}}%
  \begin{minipage}{\columnwidth}%
 \fi\fi%
 \def\FrameCommand##1{\hskip\@totalleftmargin \hskip-\fboxsep
 \colorbox{shadecolor}{##1}\hskip-\fboxsep
     % There is no \\@totalrightmargin, so:
     \hskip-\linewidth \hskip-\@totalleftmargin \hskip\columnwidth}%
 \MakeFramed {\advance\hsize-\width
   \@totalleftmargin\z@ \linewidth\hsize
   \@setminipage}}%
 {\par\unskip\endMakeFramed%
 \at@end@of@kframe}
\makeatother

\definecolor{shadecolor}{rgb}{.97, .97, .97}
\definecolor{messagecolor}{rgb}{0, 0, 0}
\definecolor{warningcolor}{rgb}{1, 0, 1}
\definecolor{errorcolor}{rgb}{1, 0, 0}
\newenvironment{knitrout}{}{} % an empty environment to be redefined in TeX

\usepackage{alltt}
\newcommand{\SweaveOpts}[1]{}  % do not interfere with LaTeX
\newcommand{\SweaveInput}[1]{} % because they are not real TeX commands
\newcommand{\Sexpr}[1]{}       % will only be parsed by R


%[xcolor=dvipsnames]
%\usetheme{default}
%\usetheme{default}
%\usetheme{Madrid}
\usetheme{Boadilla}
\usepackage{graphicx}
\usepackage{graphics}
\usepackage{epsfig}
%\usepackage{pdfpages}
\usepackage{amsmath,amssymb} % \fleqn if taken out math environments will get centered
\usepackage{tikz}
\usetikzlibrary{backgrounds}
\documentclass[11pt,red]{beamer}

%\usepackage{inc/inc}
\usepackage{amsmath,amssymb,amsfonts,dsfont,amscd}
\usepackage{graphicx,bm,booktabs,subfigure}

\usepackage{url}
\usepackage{textcomp}
\usepackage[vcentermath]{youngtab}

\usepackage{multirow}
\usepackage{epsfig}
\usepackage{latexsym}
\usepackage{amssymb}
\usepackage{amstext}
\usepackage{amsgen}
\usepackage{amsxtra}
\usepackage{amsgen}
\usepackage{amsthm}
\usepackage{color}

\usepackage{chemarr}
\usepackage[mathscr]{eucal}
\usepackage{xspace}
\usepackage{setspace}
\usepackage{booktabs}

\newcommand{\M}{\operatorname{M}}
\newcommand{\mr}{\operatorname{mr}}

% notation related to skew min rank
\newcommand{\s}{\mathcal{S}}
\newcommand{\sS}{\mathcal{S}^-}
\newcommand{\smr}{\operatorname{mr}^-}
\newcommand{\sMR}{\operatorname{MR}^-}
\newcommand{\sM}{\operatorname{M}^-}
\newcommand{\sZ}{\operatorname{Z}^-}
\newcommand{\srv}{\operatorname{r}_v^-}
\newcommand{\G}{\mathcal{G}}
\newcommand{\R}{\mathbb{R}}

%blue red
\definecolor{myBlue}{rgb}{0.1,0.1,0.45098039}
\def\Head#1{\noindent{\color{myBlue} #1}}
\newcommand{\blue}[1]{{\color{blue} #1}}
\newcommand{\red}[1]{{\color{red} #1}}
\newcommand{\normal}{\mathcal{N}}
\setbeamercovered{transparent}

%%%%%%%%%%%%%%%%%%%%%%%%%%%%%%%%%%%%%%%%%%%%%%%%%%%%%%%%%%%
%%%    FANCY QUOTES                                     %%%
%%%%%%%%%%%%%%%%%%%%%%%%%%%%%%%%%%%%%%%%%%%%%%%%%%%%%%%%%%%
%\makeatletter
%\tikzset{%
%  fancy quotes/.style={
%    text width=\fq@width pt,
%    align=justify,
%    inner sep=1em,
%    anchor=north west,
%    minimum width=\textwidth,
%  },
%  fancy quotes width/.initial={.8\textwidth},
%  fancy quotes marks/.style={
%    scale=8,
%    text=white,
%    inner sep=0pt,
%  },
%  fancy quotes opening/.style={
%    fancy quotes marks,
%  },
%  fancy quotes closing/.style={
%    fancy quotes marks,
%  },
%  fancy quotes background/.style={
%    show background rectangle,
%    inner frame xsep=0pt,
%    background rectangle/.style={
%      fill=gray!25,
%      rounded corners,
%    },
%  }
%}
%
%\newenvironment{fancyquotes}[1][]{%
%\noindent
%\tikzpicture[fancy quotes background]
%\node[fancy quotes opening,anchor=north west] (fq@ul) at (0,0) {``};
%\tikz@scan@one@point\pgfutil@firstofone(fq@ul.east)
%\pgfmathsetmacro{\fq@width}{\textwidth - 2*\pgf@x}
%\node[fancy quotes,#1] (fq@txt) at (fq@ul.north west) \bgroup}
%{\egroup;
%\node[overlay,fancy quotes closing,anchor=east] at (fq@txt.south east) {''};
%\endtikzpicture}
%\makeatother
%%%%%%%%%%%%%%%%%%%%%%%%%%%%%%%%%%%%%%%%%%%%%%%%%%%%%%%%%%%
%%%  End of fancy quotes                                %%%
%%%%%%%%%%%%%%%%%%%%%%%%%%%%%%%%%%%%%%%%%%%%%%%%%%%%%%%%%%%


%title information
\title[SASnatch]{SASnatch: Using SAS Naturally}
\subtitle{(or knitrally)}
\subtitle{\smaller{(or snatchurally)}}
\author[Mouzon]{Ian Mouzon}
\institute[Stats@ISU]{Department of Statistics\\Iowa State University }
\date[Apr 30, 2014]{April 30, 2014}



\begin{document}
% For LaTeX-Box: root = presentation-stat585.tex 
%%%%%%%%%%%%%%%%%%%%%%%%%%%%%%%%%%%%%%%%%%%%%%%%%%%%%%%%%%%%%%%%%%%%%%%%%%%%%%%%
%  File Name: presentation-stat585.rnw
%  Purpose:
%
%  Creation Date: 29-04-2014
%  Last Modified: Wed Apr 30 13:41:33 2014
%  Created By:
%%%%%%%%%%%%%%%%%%%%%%%%%%%%%%%%%%%%%%%%%%%%%%%%%%%%%%%%%%%%%%%%%%%%%%%%%%%%%%%%
%-- Set parent file



\begin{frame}
\titlepage
\end{frame}

\begin{frame}
\frametitle{What is SAS}
\framesubtitle{And why should you care}
\centerline{\includegraphics[scale=.4]{/Users/user/Desktop/SASis}} \\

\textbf{Multiple Choice:}
SAS is one of the following
\begin{itemize}
   \item[(a)] 
      SAS is a software suite developed by SAS institute for advanced analytics
   \item[(b)] 
      A frustrating and cumbersome tool that has been losing 
      its place in academia and increasingly in industry and which will not be
      missed when it disappears forever
   \item[(c)]
      A programming language that you will have to use in one or two classes
      but then all your professors after that will use R
   \item[(d)]
      All of the above
\end{itemize}
\pause The answer is (a). The other choices are all incorrect.
\end{frame}

\begin{frame}
   \frametitle{SAS: It's actually really good}
   \framesubtitle{It's better than you remember}
   \begin{itemize}
      \item SAS code that works now will still work years from now and SAS
         code that ran years ago will still run today 
         \vspace{.2cm}

      \item SAS is very well documented (something that matters 
            more as you move away from STAT 500 material),
         \vspace{.2cm}

         \item \textbf{Three of the big topics from this class}
            \begin{itemize}
               \item SAS has interactive graphics, \\
               \item built in tools for \textit{static or dynamic} web publishing, \\
               \item built in database management methods 
            \end{itemize}
         \vspace{.2cm}

         \item SAS runs of linux
   \end{itemize}
\end{frame}

%at this point do a simple example showing how to
%do an analysis in SAS
\begin{frame}[fragile]
   \frametitle{Example: Principle components}
   \begin{verbatim}
      OPTIONS LS=80;
      DATA TURTLES;
         INFILE "/home/dicook/data/johnson.and.wichern/turtles.dat";
         INPUT LENGTH WIDTH HEIGHT SEX;
      PROC PRINCOMP COV OUT=PCA1;
      VAR LENGTH WIDTH HEIGHT;
      PROC PLOT DATA=PCA1;
      PLOT PRIN2*PRIN1;
      PLOT PRIN3*PRIN1;
      PLOT PRIN3*PRIN2;
      RUN;
   \end{verbatim}
\end{frame}

\begin{frame}
   \frametitle{A few problems you may have just noticed}
   \framesubtitle{Or noticed over and over again in STAT 500}
   \begin{itemize}
      \item There is too much output.
      \begin{itemize}
         \item SAS seems to err on the side of caution and gives us more than we usually need.
      \end{itemize}
      \item The plots are underwhelming
      \begin{itemize}
         \item SAS can create very high quality plots but these don't seem to be
         \item We can always export the data sets so that we can make our plots in R, but this would be an enormous amount of trouble
      \end{itemize}
      \pause
      \item The output does not allow us to make changes.
      \begin{itemize}
         \item We can dig into the html file produced in the second example
            but this is a far cry from the ease of using knitr
      \end{itemize}
   \end{itemize}
   Most of my complaints about SAS have to do with the way it fails to work like knitr.
\end{frame}

\begin{frame}
   \frametitle{SASnatch: Using SAS in a more natural way}

   SASnatch is a package that aims to do more the following:
   \begin{itemize}
      \item Bridge the gap between SAS code and reproducible, manipulatable output
      \item Allow the programmer to use which tool they prefer at that moment, be
         it is SAS or R
   \end{itemize}

   It does this through knitr. I will show you how this works.
\end{frame}

\begin{frame}
   \frametitle{SASnatch: Using SAS in a more natural way}

   SASnatch is a package that aims to do more the following:
   \begin{itemize}
      \item Bridge the gap between SAS code and reproducible, manipulatable output
      \item Allow the programmer to use which tool they prefer at that moment, be
         it is SAS or R
   \end{itemize}

   It does this through knitr. I will show you how this works.
\end{frame}

\begin{frame}[fragile]
   \frametitle{Installation}

   SASnatch is available on github at: \vspace{.3cm}

   \verb! https://github.com/imouzon/SASnatch! \vspace{.4cm} \\

   \begin{itemize}
      \item in R it can be installed on \verb!ts3! using 
   \end{itemize}
   \verb!    >library(devtools)!
   \verb!    >install_github('SASnatch','imouzon',!
   \verb!                    arg='-l U://Documents/R/win-library/3.0')!
   \vspace{.4cm} \\

\end{frame}
\end{document}
